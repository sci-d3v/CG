\documentclass{beamer}
\usepackage{animate}
\usepackage{multimedia}
\usepackage[english,russian]{babel}

\usepackage[T2A]{fontenc}
\usepackage[utf8]{inputenc}

\usetheme{CambridgeUS}
\usecolortheme{dolphin}


\title[Введение в КГ]{Введение в компьютерную графику}
\author[Быковских Д.А.]{Быковских Дмитрий Александрович}
\date{02.09.2023}

\begin{document}
	\begin{frame}
		\titlepage
	\end{frame}
	%\section{Обзор}
	\begin{frame}{Содержание}
		\begin{itemize}
			\item 
			Структура дисциплны
			\item
			История развития компьютерной графики
			\item
			Аппаратные средства, связанные с выводом изображения
			\item 
			Библиотеки визуализации
		\end{itemize}
	\end{frame}
	\begin{frame}{Что такое компьютерная графика?}
		Компьютерная графика --- одно из направлений информационных технологий, которое занимается созданием, редактированием и визуализацией \textbf{графических изображений}.
		
		Спектр применений:
		
		\begin{columns}
			
			\begin{column}{0.5\textwidth}
				\begin{itemize}
				\item
				Разработка игр (Виртуальный мир, персонажи, эффекты)
				
				\item
				Визуализация данных (диаграммы, графики)
				\item
				Дизайн (Логотипы, баннеры, упаковки, интерфейсы)
				\item
				Симуляция и моделирование (Создание виртуальных сред для тестирования и исследования различных сценариев)
				
			\end{itemize}
			\end{column}
			\begin{column}{0.5\textwidth}
				\begin{itemize}
					\item
					Анимация (фильмы, видеоролики, реклама)
				\item
				Медицинская визуализация (модели органов, тканей и других частей тела, )
				\item
				Архитектурное проектирование (модели жилых районов, зданий, интерьеров)
				
				\item
				Редактирование фотографий и видеороликов (улучшение качества изображений)
			\end{itemize}
			\end{column}
		
		\end{columns}
		
		\if 0
		3D-моделирование и анимация: Создание трехмерных моделей объектов, персонажей и сцен, а также их анимация для использования в фильмах, играх, визуализации архитектурных проектов и многих других областях.
		
		2D-графика и иллюстрации: Создание двухмерных изображений, иллюстраций, рисунков, артов и графических элементов для книг, журналов, рекламы и дизайна.
		
		Визуализация данных: Преобразование сложных данных в наглядные и понятные визуальные формы, такие как графики, диаграммы, инфографика и карты.
		
		Компьютерная анимация и эффекты: Создание анимаций, спецэффектов и визуальных элементов для фильмов, рекламы, игр и других медийных продуктов.
		
		Графический дизайн: Разработка дизайна логотипов, брендинга, упаковки, веб-сайтов, приложений и других визуальных элементов.
		
		Виртуальная реальность (VR) и дополненная реальность (AR): Создание виртуальных и дополненных миров, в которых пользователи могут взаимодействовать и иметь визуальный опыт.
		
		Компьютерное моделирование: Создание компьютерных моделей и симуляций для изучения поведения систем, процессов и явлений в науке, инженерии и других областях.
		
		Медицинская визуализация: Создание графических изображений и моделей для обучения, диагностики и планирования медицинских процедур.
		
		Архитектурная визуализация: Создание виртуальных моделей зданий и помещений для архитекторов и дизайнеров.
		
		Ретушь и обработка фотографий: Использование графических инструментов для улучшения и изменения фотографий.
		
		Генеративное искусство: Использование алгоритмов и программирования для создания уникальных искусственных изображений и анимаций.
		\fi
		
	\end{frame}
	%https://iasl.uni-muenchen.de/links/GCA-IV.2e.html
	%https://www.timetoast.com/timelines/6248910a-e70f-4a32-b27c-e13ef9c355dc
	%https://web.archive.org/web/20070405181508/http://accad.osu.edu/\%7Ewaynec/history/lesson2.html
	%https://web.archive.org/web/20051103164835/http://accad.osu.edu/~waynec/history/lesson2.html
	
	История компьютерной графики насчитывает несколько десятилетий и прошла через ряд важных этапов и достижений. Вот краткий обзор этой истории:
	
	1950-е годы: Ранние исследования в области компьютерной графики начались в 1950-х годах. В 1950 году Иван Сазерленд (Ivan Sutherland) создал устройство "Скетчпад" (Sketchpad), позволяющее рисовать графику с помощью светового пера и дисплея.
	
	1960-е годы: В 1963 году Сазерленд представил прототип первого компьютерного графического устройства для рисования 3D-изображений, названного "TX-2 Sketchpad". Этот момент считается началом трехмерной компьютерной графики.
	
	1970-е годы: В 1970-х годах были разработаны первые программы для создания и редактирования графики, такие как "SuperPaint" и "Computer Graphics System" на основе аппаратных решений. Были также созданы первые алгоритмы заполнения и отсечения для растровой графики.
	
	1980-е годы: В это десятилетие произошли значительные прорывы в компьютерной графике. Были разработаны первые графические интерфейсы пользователя (GUI), например, интерфейс для компьютера Apple Macintosh. Также были созданы первые 3D-моделирование и анимационные программы.
	
	1990-е годы: С распространением персональных компьютеров и улучшением аппаратного обеспечения появились новые возможности для компьютерной графики. Были разработаны 3D-игры, программы для редактирования фотографий и видеороликов.
	
	2000-е годы: Развитие 3D-графики продолжилось, и стали доступны более мощные инструменты для создания сложных 3D-моделей и анимаций. Также стали популярными виртуальная и дополненная реальность.
	
	2010-е годы: Продолжалось усовершенствование компьютерной графики, включая разработку более реалистичных графических движков для видеоигр, прорывы в области визуализации данных и рост популярности анимации и цифрового искусства.
	\begin{frame}
		
	\end{frame}

\begin{frame}
	
\end{frame}
\begin{frame}
	
\end{frame}
	
\end{document}