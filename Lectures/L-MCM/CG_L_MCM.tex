\documentclass{beamer}
\usepackage{animate}
\usepackage{multimedia}
\usepackage[english,russian]{babel}

\usepackage{pgfpages}
\setbeameroption{show notes on second screen}
%https://tug.ctan.org/macros/latex/contrib/beamer/doc/beameruserguide.pdf

\usepackage[T2A]{fontenc}
\usepackage[utf8]{inputenc}

\setbeamertemplate{caption}[numbered]

\usetheme{CambridgeUS}
\usecolortheme{dolphin}


\title[МКМ]{Монте-Карло метод}
\author[Быковских Д.А.]{Быковских Дмитрий Александрович}
\date{16.09.2023}

\begin{document}
	\begin{frame}
		\titlepage
	\end{frame}
	%\section{Обзор}
	\begin{frame}{Содержание}
		\begin{itemize}
			\item 
			Определение, понятия, история
			\item
			Генератор псевдослучайных чисел
			\item
			Estimator
			\item
			Квази МКМ
			\item 
			Скорость сходимости
			\item
			MISER метод и др.
			\item 
			Применение метода в компьютерной графике
		\end{itemize}
	\end{frame}
	\begin{frame}{Краткая справка}{Monte-Carlo Method (MCM)}
		Монте-Карло метод (ММК) --- численный метод решения математических задач при помощи моделирования случайных величин.
		%Ссылка на Соболь Монте-Карло Метод Популярные лекциии по математике
		
		Название происходит от города Монте-Карло в княжестве Монако, знаменитого своим игорным домом. 
		
		\note{Рулетка --- простейший прибор получения случайной величины. }
		
		Датой рождения: 1949 г., когда появилась статья под названием The Monte Carlo method.
		Создатели: Дж. Нейман и С. Улам
		%Metropolis, N., Ulam, S. The Monte Carlo Method, — Journal of the American Statistical Association 1949 44 № 247 335—341.
		
		Теоритечкская основа известна давно.
		
		\note{ \footnotesize
			
		Метод Монте-Карло — [84; 85].
		
		Первая работа по использованию вероятностного метода была опубликована А. Холлом [86] в 1873 году именно при организации стохастического процесса экспериментального определения числа $\pi$ путем бросания иглы на лист линованной бумаги. Идея такого эксперимента возникла у Ж.Л.Л. Бюффона для
		вычисления числа $\pi$ в 1777 году.
		
		Бурное развитие и применение методов статистического моделирования (Монте-Карло) в различных областях прикладной математики началось с середины прошлого столетия. Это было связано с решением качественно новых задач, возникших при исследовании новых процессов. Одним из первых, кто применил ММК для моделирования траекторий нейтронов был Дж. фон Нейман. Первая работа с систематическим изложением была опубликована в 1949 году Н.К. Метрополисом и С.М. Уламом [87]. Метод Монте-Карло применялся для решения линейных интегральных уравнений, в котором решалась задача о прохождении нейтронов через вещество.
		
		%84. Соболь, И. М. Численные методы Монте-Карло / И. М. Соболь. –– М. : НАУКА, 1973. –– 312 с.
		%85. Марчук, Г. И. Метод Монте-Карло в атмосферной оптике / Г. И. Марчук, Г. А. Михайлов. –– М. : Наука, 1976. –– 145 с.
		%86. Hall, A. On an experiment determination of π / A. Hall // Messeng. Math. –– 1873. –– No. 2. –– P. 113––114. 108
		%87. Metropolis, N. The Monte-Carlo method / N. Metropolis, S. Ulam // J. Amer. Stat. Assoc. –– 1949. –– Vol. 44, no. 247. –– P. 335––341.
	}
	
 	\end{frame}
	\begin{frame}{Краткая справка}
		Особенности метода:
		\begin{enumerate}
			\item 
			Простая структура вычислительного алгоритма, т.е. необходимо описать действие одного шага.
			А потом множество шагов усреднить.
			Метод статистических испытаний.
			\item
			Ошибка вычислений, как правило, пропорциональная 
			$$\sqrt{\frac{D}{N}},$$ 
			где $D$ --- некоторая постоянная, а $N$ --- число испытаний.
			%проанализировать это.
			Метод эффективен, когда высокая точность не сильно важна.
			
		\end{enumerate}
		
		Задачи решаемые с помощью ММК:
		\begin{enumerate}
			\item
			Любой процесс, на протекание которого влияют случайные факторы.
			\item 
			Для любой задачи можно искусственно придумать вероятностную модель или даже несколько.
		\end{enumerate}		
	\end{frame}
	
	\begin{frame}{Вычисление площади фигуры}{Пример}
		
		\note{Стрелок плохо стрелят или хорошо, как это скажется на результатах?}
		
	\end{frame}
	
	\begin{frame}{Вычисление площади определенного интеграла}{Пример}
		Постановка задачи:
		\[
		I = \int_{a}^{b} f(x) dx
		\]
		
		Метод центральных прямоугольников
		\[
		I 
		\approx 
		\frac{1}{N} \sum_{i=1}^{n} f(x_i) \Delta x 
		=
		\frac{b-a}{N} \sum_{i=1}^{n} f(x_i)
		,
		\]
		где $\sum_{i=1}^{n} \Delta x = b - a$.
		\[
		I
		=
		(b - a) M [f(x)]
		\]
		Пусть плотность распределения $\rho(x) = \frac{1}{b-a}$. Тогда можно получить Estimator следующим образом:
		\note{Estimator --- правило для вычисления статистической оценки, определяющее скорость сходимости.}
		\[
		I
		= 
		\int_{a}^{b} \frac{f(x)}{\rho(x)} \rho(x) dx
		= 
		M \Bigg[\frac{f(x)}{\rho(x)}\Bigg] 
		\approx
		\frac{1}{N} \sum_{i=1}^{n} \frac{f(\xi_i)}{\rho(\xi_i)}
		.
		\]
	\end{frame}
	
	\begin{frame}{Схема анализа задачи}
		\begin{enumerate}
			\item
			Модель задачи.
			\item
			Алгоритм общий.
			\item
			Случайная величина.
			\item
			Точность.
			
			когда знаешь, точный результат
			
			когда не знаешь, точный результат
			\item
			Погрешность.
			\item
			Как уменьшить погрешность?
		\end{enumerate}
	\end{frame}
			
\end{document}