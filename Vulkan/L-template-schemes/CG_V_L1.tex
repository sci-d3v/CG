\documentclass{beamer}
% \usepackage{animate}
\usepackage{multimedia}
\usepackage[english,russian]{babel}

\usepackage{pgfpages}
\setbeameroption{show notes on second screen}
%https://tug.ctan.org/macros/latex/contrib/beamer/doc/beameruserguide.pdf

\usepackage[T2A]{fontenc}
\usepackage[utf8]{inputenc}

\setbeamertemplate{caption}[numbered]

\usetheme{CambridgeUS}
\usecolortheme{dolphin}

\usepackage{pgfplots}
% \pgfplotsset{width=10cm,compat=1.17} % Установка ширины графика и совместимости

\usepackage{tikz}
% Библиотеки для создания геометрических фигур и стрелок
\usetikzlibrary{shapes.geometric, arrows} 
\tikzstyle{block} = [rectangle, rounded corners, minimum width=3cm, minimum height=1cm, text centered, draw=black, fill=blue!30]
\tikzstyle{arrow} = [thick,->,>=stealth]


\title[Введение в КГ]{Введение в компьютерную графику}
\author[Быковских Д.А.]{Быковских Дмитрий Александрович}
\date{07.09.2024}

\begin{document}
	\begin{frame}
		\begin{tikzpicture}[node distance=1.5cm]
			% Nodes
			\node (application) [block,minimum height=2.5cm] {Application};
			\node (command_buffer_1) [block, maximum height=1.0cm, right of=application, yshift=-1.75cm, align=center] {Command\\buffer};
			\node (command_buffer_2) [block, right of=application, yshift=1.75cm, align=center] {Command\\buffer};
			% Lines
			\draw [arrow] (application) -- (command_buffer_1);
			\draw [arrow] (application) -- (command_buffer_2);
		\end{tikzpicture}
	\end{frame}




	\begin{frame}
		\titlepage
	\end{frame}
	%\section{Обзор}
	\begin{frame}{Содержание}
		\begin{itemize}
			\item 
			Общая информация о компьютерной графике
			\item
			Некоторые факты из истории компьютерной графики
			\item
			Аппаратные средства, связанные с выводом изображения
			\item 
			Библиотеки визуализации
		\end{itemize}
	\end{frame}

	\begin{frame}
	\begin{tikzpicture}
    \begin{axis}[
        xlabel={$x$}, % Подпись оси X
        ylabel={$y$}, % Подпись оси Y
        title={График функции $y = x^2$}, % Заголовок графика
        grid=major % Включение сетки
    ]
    \addplot[domain=-2:2, thick, color=blue] {x^2}; % Построение функции y = x^2
    \end{axis}
\end{tikzpicture}
\end{frame}

\begin{frame}
	\begin{tikzpicture}[node distance=1.5cm]

    % Узлы
    \node (start) [block] {Начало};
    \node (process1) [block, below of=start] {Процесс 1};
    \node (decision) [diamond, aspect=2, minimum width=3cm, minimum height=1cm, text centered, draw=black, fill=red!30, below of=process1] {Решение};
    \node (process2a) [block, below of=decision, xshift=-2.5cm] {Процесс 2а};
    \node (process2b) [block, below of=decision, xshift=2.5cm] {Процесс 2б};
    \node (end) [block, below of=process2a, xshift=2.5cm] {Конец};

    % Стрелки
    \draw [arrow] (start) -- (process1);
    \draw [] (process1) -- (decision);
    \draw [arrow] (decision) -- node[anchor=south] {Да} (process2a);
    \draw [arrow] (decision) -- node[anchor=south] {Нет} (process2b);
    \draw [arrow] (process2a) -- (end);
    \draw [arrow] (process2b) -- (end);

	\end{tikzpicture}
\end{frame}

\if 0
\begin{columns}
	
	\begin{column}{0.5\textwidth}
		\begin{itemize}
			\item
			
		\end{itemize}
	\end{column}
	\begin{column}{0.5\textwidth}
		\begin{itemize}
			\item
		\end{itemize}
	\end{column}
	
\end{columns}
\fi
	
\end{document}
