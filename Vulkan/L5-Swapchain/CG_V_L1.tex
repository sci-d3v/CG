\documentclass{beamer}
% \usepackage{animate}
\usepackage{multimedia}
\usepackage[english,russian]{babel}

\usepackage{pgfpages}
\setbeameroption{show notes on second screen}
%https://tug.ctan.org/macros/latex/contrib/beamer/doc/beameruserguide.pdf

\usepackage[T2A]{fontenc}
\usepackage[utf8]{inputenc}

\setbeamertemplate{caption}[numbered]

\usetheme{CambridgeUS}
\usecolortheme{dolphin}


\title[Введение в КГ]{Введение в компьютерную графику}
\author[Быковских Д.А.]{Быковских Дмитрий Александрович}
\date{07.09.2024}

\begin{document}
	\begin{frame}
		\titlepage
	\end{frame}
	%\section{Обзор}
	\begin{frame}{Содержание}
		\begin{itemize}
			\item 
			Общая информация о компьютерной графике
			\item
			Некоторые факты из истории компьютерной графики
			\item
			Аппаратные средства, связанные с выводом изображения
			\item 
			Библиотеки визуализации
		\end{itemize}
	\end{frame}
	
	\begin{frame}
		swapchain?
		Vulkan API has two types of resource: buffers and images

		Vulkan images represent contiguous texture data in 1D/2D/3D form. These images are primarily used as either an attachment or texture:
			Attachment: The image is attached to the pipeline for the framebuffer's color or depth attachment and can also be used as an auxiliary surface for multipass 	processing purposes
			Texture: The image is used as a descriptor interface and shared at the shader stage (fragment shader) in the form of samplers

			the image is created by specifying a number of bitwise fields indicating the kind of image usage, such as color attachment, depth/stencil attachment, a sampled image in a shader, image load/store, and so on. 
			In addition, you need to specify the tiling information (linear or optimal) for the image. This specifies the tiling or swizzling layout for the image data in memory.

			The notion of texture
			Image: An image represents the texture object in Vulkan.
			contains metadata for memory requirements such as the format, size, and type (sparse
			map, cube map, and so on). multiple images, based on the mipmap level or array layers

			An image layout is an implementation-specific way to store image texture information in a grid coordinate representation in the image memory.

			Image view: Images cannot be used directly for reading and writing purposes by API calls or pipeline shaders; instead, image views are used. It not only acts like an interface to the image object, but it also provides the metadata that is used to represent a continuous range of image sub-resources.

	\end{frame}

\if 0
\begin{columns}
	
	\begin{column}{0.5\textwidth}
		\begin{itemize}
			\item
			
		\end{itemize}
	\end{column}
	\begin{column}{0.5\textwidth}
		\begin{itemize}
			\item
		\end{itemize}
	\end{column}
	
\end{columns}
\fi
	
\end{document}
