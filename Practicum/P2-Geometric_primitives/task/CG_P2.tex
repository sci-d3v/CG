\documentclass[a4paper,12pt]{article}

\usepackage{geometry}
\geometry{left=2cm, right=2cm, top=2cm, bottom=2cm}

\usepackage[english,russian]{babel}
\usepackage[T2A]{fontenc}
\usepackage[utf8]{inputenc}

\usepackage{amsmath,amsfonts,amssymb,amsthm,mathtools}
\usepackage[]{hyperref}

\usepackage{float}

\begin{document}
\fontsize{14pt}{16pt}\selectfont
\begin{center}
    \textbf{{\Large Компьютерная графика}}
    
    \textbf{{\large Лабораторная работа №2. }}
    
    \textbf{{\large Построение изображения с помощью графических примитивов}}
    \end{center}
    
    
    \textbf{Задание}
    
    Изучите проект программы и создайте изображение аппликации согласно вашему варианту, модифицировав исходный проект.
    
    
    
    \textbf{Ход выполнения работы}
    
    \textbf{Шаг 1.} \textit{Настройка проекта и запуск проекта в ИСР}
    
    Директория "CG/Practicum" содержит различные проекты. Чтобы выполнить текущее задание, откройте в отдельном окне проект, расположенный \textquotedbl CG/Practicum/P2-Geometric\_primitives\textquotedbl, с помощью настроенной ИСР.

    Чтобы установить и настроить ИСР, можно воспользоваться инструкцией, расположенной в директории 
    \textquotedbl CG/Practicum/P0-Getting\_started\textquotedbl .

    %Чтобы скомпилировать проект, необходимо настроить CMake, с помощью которого выполняется автоматическая сборка текущего проекта.
    %Нажмите комбинацию клавиш Ctrl+Shift+P и выполните команду "CMake: Select a Kit"; в зависимости от операционной системы выберете файл с настройками, например, для ОС Windows --- "gcc and g++".

    % Для просмотра результата нажмите сочетание клавиш Ctrl+Shift+P и выполните команду "Show glslCanvas", которая запускает плагин glsl-сanvas и отображает результат работы пиксельного шейдера на экране.

    % В директории с заданием найдите и скопируйте исходный проект с названием "CG\_P2" в директорию с другими проектами, расположенными на рабочем компьютере. Например, в дистрибутивах Linux проекты расположены в директории "/home/CLionProjects".
    
    % Запустите и откройте проект с помощью интегрированной среды разработки CLion. 
    
    % Скомпилируйте проект, нажав комбинацию клавиш "Shift + F10".
    
    \textbf{Шаг 2.} \textit{Изучение проекта и необходимой литературы.}
    
    Проект собирается из следующих основных файлов:
    
    1. Файл конфигурации, предназначенный для сборки проекта, имеющий название \textquotedbl CMakeCache.txt\textquotedbl и расположенный в рабочей директории проекта.
    
    2. Файл с исходным кодом программы, имеющий название "main.c" и расположенный в рабочей директории проекта.
    
    3. Файлы с различными типами шейдеров, имеющие расширение \textquotedbl glsl\textquotedbl и расположенные в папке \textquotedbl shaders\textquotedbl.
    
    
    
    Изучите указанную литературу и сделайте краткий конспект изученного материала, как минимум содержащий развернутые ответы на следующие контрольные вопросы:
    \begin{enumerate}
    \item Чем отличаются режимы \textquotedbl Core-Profile\textquotedbl от \textquotedbl Compatibility-Profile\textquotedbl. Для чего они нужны?
    \item Как расшифровывается аббревиатуры GLFW и GLEW?
    \item Каково назначение следующих библиотек: GL, GLEW и GLFW? 
    \item Как отличить, какая функция в файле с исходным кодом к какой библиотеке относится?
    \item Что такое тело функции, прототип функции и вызов функции?
    \item Что требуется добавить в проект, чтобы использовать функции и другие возможности, например, из библиотеки GLEW?
    \item Как устроена шейдерная программа? Что необходимо сделать для его успешной работы?
    \item Какие шейдеры используются в проекте, в чем их отличие?
    \item Как расшифровывается аббревиатура VAO, для чего он нужен?
    \item Как расшифровывается аббревиатура VBO, для чего он нужен?
    \item Какие типы графических примитивов вы знаете?
    \item Сколько раз рисуется картинка на экране?
    \item Что такое буфер кадра?
    \end{enumerate}
    
    
    
    Список основной литературы:
    
    \begin{enumerate}
        \item 
        Вершинные массивы, задание атрибутов при помощи вершинных массивов. Работа с шейдерами // Боресков А.В. Программирование компьютерной графики. Современный OpenGL. - М.: ДМК Пресс, 2019. - 372 с.
        \item 
        \href{https://habr.com/ru/post/310790/}{Уроки 1.1 - 1.5. // LearnOpenGL (автор оригинала: Joey de Vries; автор перевода: Megaxela)} 
        
        \item
        Глава 6. Графическое программирование // Перемитина Т.О. Компьютерная графика. - Томск: ТУСУР, 2012. - 144 с.
        
        \end{enumerate}

    Список дополнительной литературы:
    
    \begin{enumerate}

    \item 
    \href{https://www.khronos.org/registry/OpenGL/specs/gl/}{GLSL. Language Specification}
    
    \item
    \href{https://www.glfw.org/docs/latest/quick.html}{GLFW: Getting started}

    \end{enumerate}
    
    
    \textbf{Шаг 3.} \textit{Построение изображения согласно варианту.}
    
    Согласно вашему варианту выберите аппликацию, которую вам необходимо нарисовать, в папке \textquotedbl CG/Practicum/P2-Geometric\_primitives/task/tests\textquotedbl. Каждая аппликация пронумерована, т.е. это и есть вариант задания.
    
    Создайте схематичное изображение той аппликации, которую вам необходимо нарисовать, т.е. разметьте на листочке основные вершины объектов и/или их частей с указанием их координат. Обратите внимание на повторяющиеся объекты, которые можно дублировать. Например, собрать объект из повторяющихся частей, как это показано в примере.
    
    Используя, полученные знания из предыдущих шагов и прошлого задания, модифицируйте файл с исходным кодом и, если необходимо, файлы с шейдерным кодом так, чтобы добиться желаемого результата.
\end{document}

