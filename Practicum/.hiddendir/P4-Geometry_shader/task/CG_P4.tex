\documentclass[a4paper,12pt]{article}

\usepackage{geometry}
\geometry{left=2cm, right=2cm, top=2cm, bottom=2cm}

\usepackage[english,russian]{babel}
\usepackage[T2A]{fontenc}
\usepackage[utf8]{inputenc}

\usepackage{amsmath,amsfonts,amssymb,amsthm,mathtools}
\usepackage[]{hyperref}

\usepackage{float}

\begin{document}
\fontsize{14pt}{16pt}\selectfont
\begin{center}
    \textbf{{\Large Компьютерная графика}}
    
    \textbf{{\large Лабораторная работа №4. }}
    
    \textbf{{\large Построение изображения с помощью геометрического шейдера}}
    \end{center}
    
    
    \textbf{Задание}
    
    Изучите проект программы и создайте изображение (фрактал) согласно вашему варианту, модифицировав исходный проект.
    
    \textbf{Ход выполнения работы}
    
    \textbf{Шаг 1.} \textit{Настройка проекта}
    
    Директория "CG/Practicum" содержит различные проекты. Чтобы выполнить текущее задание, откройте в отдельном окне проект, расположенный \textquotedbl CG/Practicum/P4-Geometry\_shader\textquotedbl, с помощью настроенной ИСР.

    Чтобы установить и настроить ИСР, можно воспользоваться инструкцией, расположенной в директории 
    \textquotedbl CG/Practicum/P0-Getting\_started\textquotedbl .
    
    \textbf{Шаг 2.} \textit{Изучение проекта и необходимой литературы}
    
    Проект собирается из следующих основных файлов:
    
    \begin{enumerate}
        \item Файл конфигурации, предназначенный для сборки проекта, имеющий название \textquotedbl CMakeCache.txt\textquotedbl~и расположенный в рабочей директории проекта.
        \item Файл с исходным кодом программы, имеющий название \textquotedbl main.cpp\textquotedbl~и расположенный в рабочей директории проекта.
        \item Файлы с различными типами шейдеров, имеющие расширение \textquotedbl glsl\textquotedbl~и расположенные в папке \textquotedbl shaders\textquotedbl.
    \end{enumerate}
    
    Изучите указанную литературу и сделайте краткий конспект изученного материала, как минимум содержащий развернутые ответы на следующие контрольные вопросы:

    \begin{enumerate}
        \item 
        В чем заключается назначение геометрического шейдера?
        \item 
        Какие типы примитивов можно указать в качестве входных и выходных данных в геометрическом шейдере?
        \item 
        Какие встроенные переменные используются в геометрическом шейдере?
        \item 
        Каким образом передается цвет по графическому конвейеру от одного шейдера к другому?
        \item 
        Какие команды используются для создания вершинных данных в геометрическом шейдере?
    \end{enumerate}

    Список основной литературы:
    \begin{enumerate}
        \item \href{https://habr.com/ru/articles/350782/}{Урок 4.9 — Геометрический шейдер // LearnOpenGL (автор оригинала: Joey de Vries; автор перевода: Megaxela)}
        \item \href{https://www.khronos.org/opengl/wiki/Geometry_Shader}{Geometry Shader}
        \item \href{https://www.khronos.org/opengl/wiki/Rendering_Pipeline_Overview}{Rendering Pipeline Overview}
        \item \href{https://ravesli.com/urok-27-geometricheskie-shejdery-v-opengl/}{Урок №27. Геометрические шейдеры в OpenGL}
    \end{enumerate}
   
    \textbf{Шаг 3.} \textit{Построение изображения согласно варианту}
    
    Выберете фрактал, который вам необходимо программно нарисовать, согласно вашему варианту из списка, расположенного в файле \textquotedbl CG/Practicum/P4-Geometry\_shader/task/tests.txt\textquotedbl.

    Используя, полученные знания из предыдущих шагов и прошлого задания, модифицируйте файлы с исходным кодом так, чтобы в файле с исходным кодом и геометрическом шейдере был расположен алгоритм построения фрактала.
    
\end{document}
