\documentclass[a4paper,12pt]{article}

\usepackage[english,russian]{babel}
\usepackage[T2A]{fontenc}
\usepackage[utf8]{inputenc}

\usepackage{amsmath,amsfonts,amssymb,amsthm,mathtools}
\usepackage[]{hyperref}

%\usepackage{amsthm}
\theoremstyle{plain}
\newtheorem{theorem}{Теорема}
\newtheorem{lemma}{Лемма}

\include{preamble.tex}
%
\begin{document}
    \begin{center}
    \textbf{{\Large Компьютерная графика}}
    
    \textbf{{\large Лабораторная работа №1. }}
    
    \textbf{{\large Построение изображения флага с помощью пиксельного шейдера}}
    \end{center}
    
    
    \textbf{Задание}
    
    Изучите исходный код программы, написанный на языке GLSL, и создайте изображение флага согласно вашему варианту задания с помощью пиксельного (фрагментного) шейдера.
        
    \textbf{Ход выполнения работы}
    
    \textbf{Шаг 1. }\textit{Настройка окружения.} 
        
    Откройте файл с расширением glsl, расположенный в папке \textquotedbl example\textquotedbl, в ИСР (IDE), например, VS Code, VS Codium или Code-OSS (см. задание P0 с инструкцией по установке). 
    
    Для просмотра результата нажмите сочетание клавиш Ctrl+Shift+P и выполните команду \textquotedbl Show glslCanvas\textquotedbl, которая запускает плагин glsl-сanvas и отображает результат работы пиксельного шейдера на экране.
    
    \textbf{Шаг 2.} \textit{Изучение языка GLSL.} 
    
    Изучите указанную литературу и сделайте краткий конспект изученного материала, как минимум содержащий развернутые ответы на следующие контрольные вопросы:
    \begin{enumerate}
    \item Что такое шейдеры? Какие виды шейдеров вы знаете? Каково их назначение?
    \item Как расшифровывается аббревиатура GLSL?
    \item На чем основан язык GLSL, в чем заключается его особенность?
    \item Какие типы встроенных переменных в язык GLSL имеются? В чем их преимущество по сравнению со встроенными переменными языка Си?
    \item Что такое uniform-переменные? Для чего они нужны?
    \item Для чего нужна переменные gl\_FragCoord и gl\_FragColor?
    \item Существуют ли встроенные функции? Если да, то приведите примеры таких функций.
    \end{enumerate}
    
    
    
    Список основной литературы:
    \begin{enumerate}
    \item 
    Приложение. Язык GLSL. // Боресков А.В. Программирование компьютерной графики. Современный OpenGL. - М.: ДМК Пресс, 2019. - 372 с.
    \item 
    \href{https://thebookofshaders.com/?lan=ru}{Главы: Введение и Алгоритмическое рисование. // The Book of Shaders (авторы: Патрицио Гонзалес Виво и Джен Лав)} 
    
    \end{enumerate}
    
    
    Список дополнительной литературы:
    \begin{enumerate}
    
    \item 
    \href{https://habr.com/ru/post/313380/
    }{LearnOpenGL. Урок 1.5 — Shaders}
    
    \item \href{https://webglfundamentals.org/webgl/lessons/ru/webgl-shaders-and-glsl.html}{Шейдеры и GLSL}
    
    \item 
    \href{https://www.khronos.org/registry/OpenGL/specs/gl/}{GLSL. Language Specification}
    
\end{enumerate}
    
    \textbf{Шаг 3.} \textit{Построение изображения флага страны согласно варианту.}
    
    Определите флаг страны для воспроизведения в соответствии с вашим индивидуальным вариантом (см. директорию tests). Важно точно воспроизвести цветовую палитру, габариты флага, а также соблюсти правильные пропорции и соотношения всех декоративных элементов (орнаментов) флага. Вам необходимо найти или создать схему флага, а потом использовать ее для построения изображения.
    
    Еще раз, внимательно изучите файлы с содержимым папки 
    \textquotedbl example\textquotedbl, 
    в которой представлен образец качественно выполненной работы по созданию флага Южной Кореи.
    
    Применив знания, полученные на предыдущих шагах, внесите необходимые изменения в код фрагментного шейдера для достижения требуемого результата.
    Дополнительные материалы для изучения, расположены в папке 
    \textquotedbl additional\_examples\textquotedbl.
\end{document}

